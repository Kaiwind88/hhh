\chapter{Wind Gust Detection and Impact Prediction for Wind Turbines}
%% the whole paper

\section{Methodology}
\subsection{Definition of spatial gusts}
The spatial gust studied in the present work is defined mathematically in this section. The scale of the gust of interest ranges from 1D to 10D (D is the diameter of a wind turbine. $D=100~m$ is used, hereinafter). The magnitude of the wind speed fluctuation of a gust region, $|\boldsymbol{v}'|$, should be 1.5 times larger than the standard deviation $\sigma$ of the wind speed over the wind field, \ie$|\boldsymbol{v}'|>1.5\sigma$ and $\boldsymbol{v}'=\boldsymbol{v}-\Bar{\boldsymbol{v}}$, where $\boldsymbol{v}$ is the local velocity vector and $\Bar{\boldsymbol{v}}$ is the mean wind velocity. The gust regions should also have good temporal coherency and local spatial connectivity around their centers. Gusts are assumed to advect roughly along the mean streamline, and the major structure should preserve during traveling.

\subsection{Data information and wind field retrieval}
The wind field utilized in this work was retrieved by a new proposed two-dimensional variational analysis method (2D-VAR) from the measurements collected by a Lockheed Martin Coherent Technologies (LMCT) WindTracer\textsuperscript{\textregistered} Doppler lidar (Louisville, CO, USA) during 25–27 June 2014, at Tehachapi, California \textcolor{yellow}{[12]}. The specification of the Doppler lidar is listed in Table 1. The lidar was located on a hill (1450 m above sea level (ASL)) at a wind farm near Tehachapi City, which is at an altitude of 1220 m (ASL). The data was collected in a horizontal plane at 1453 m ASL which included the height of the lidar system (3 m).

\begin{table}
\caption{{Specifications of the Doppler lidar.}}
\centering
\label{table:lidarspec}
\begin{tabular}{|l|l|l|}
\hline
Parameters    & Settings \\ \hline
Wavelength & $1.6 \mu m$              \\ \hline
Pulse energy & $2~mJ$      \\ \hline
Pulse repetition frequency & $750\mathit{Hz}$     \\ \hline
Range resolution & $100~m$        \\ \hline
Blind zone & $436~m$          \\ \hline
Max range &	$10~km$ \\ \hline
\end{tabular}%
\end{table}

The 2D-VAR method used in the present work is based on a variational parameter identification formulation \textcolor{yellow}{[12]}. The method involves finding the best fit 2D wind velocity vector ($\boldsymbol{X}$) which minimizes a cost function: $J(\boldsymbol{X})=\frac{1}{\Omega}\int\sum W_i C_i^2d\Omega$, where $\boldsymbol{W}$ is a pre-defined weight matrix which determines the relative importance of the terms in the cost function. The constrains, $C$, are functions of the wind vector $\boldsymbol{X}$, and are comprised of radial velocity equation, tangential velocity equation for low elevation angles and the advection equation. And $\Omega$ represents the analysis domain. A quasi-Newton method is implemented for the minimization. This retrieval algorithm has the advantage of preserving local structures in complex flows while being computationally efficient with possible real-time applications.\\
The retrieval algorithm was used to convert the measured radial velocities to two-dimensional (2D) wind field in Cartesian coordinate in a domain of size $6~km \times 4~km$. A sample contour plot of the retrieved wind field is shown in Figure 1. The temporal resolution of the retrieved results is $30~s$ as determined by the lidar scanning pattern, and the size of the spatial grid is $80~m\times80~ m$ as specified by the retrieval algorithm. Note that even though the 2D-VAR method was used in the present work, the proposed detecting and tracking algorithm can be applied to any 2D wind field retrieved from any algorithm or obtained from any experimental instrument, given sufficient spatial and temporal resolution.
\subsection{Data preprocessing}
Data quality control was first performed in the wind field. The dark red regions at the northeast and southwest corners in \todo{Figure 1} were removed. The reason the data at the two corners were removed is two-fold: First, the hills were treated as hard targets, and therefore, due to possible contamination in the range-gates immediately in front of the hard targets, the data at the hills and 1–2 range gates before the hills were removed before running the retrieval algorithm. Second, the retrieval algorithm requires the data to be treated with a Gaussian filter to minimize the effect of noise on the gradients for the advection term in the cost function. Therefore, the Gaussian filter could not be applied at the boundaries of the lidar scan due to the missing data, which caused the artifacts at the corners in Figure 1, so the results 1–2 more range gates before the hills were removed.\\
Additionally, data with very high magnitudes above $30~ m/s$ were rejected since they were judged to be spurious for this dataset. While a small amount of still valid data might be filtered out, we found empirically $30~ m/s$ could be a reasonable trade-off value between removing most of the noises and keeping sufficient valid data points.\\
Moreover, the spatial resolution of the dataset needs to be considered. The current grid size is $80~m$, implying any structures smaller than the grid size will be smoothed. Only very few data points were left for atmospheric structures with scales of a few hundred meters. Because the gust extraction process in the next step (Section 2.4)\todo{section2.4} is based on the wind speed at the data points and the size and shape of the gust patch, the truncation process may cause unnatural-looking contours of the extracted gusts. Because the shape of the small-scale gusts is reduced to few pixels connected only at the vertices, some small patches could be easily neglected during the boundary tracing process. Therefore, to avoid losing many valid small-scale patches, linear interpolation using Delaunay triangulation was used to keep the shape of the gust contour to some extent \todo{[13]}[13]. Delaunay triangulation is a method that triangulates the discrete points in a plane such that no discrete point is inside the circumcircle of any triangle in the triangulation of the points. In the Delaunay triangulation, same weights were assigned to the vertices of the triangular. The wind field after the interpolation is shown in \todo{Figure}Figure 2. However, the interpolated boundaries should not be interpreted as better approximations to true boundaries than the coarse grid. Ideally, one may avoid the interpolation step, if the lidar range-gate size were significantly smaller.
\subsection{Detection of gust patches}
Peak over threshold (POT) method was used to extract gust regions from the wind field [6]. The POT method is to detect any wind event that has an amplitude over a predefined threshold. Here, we used the POT method for both up-crossing and down-crossing the threshold. The POT method removes the data points not satisfying the velocity threshold mentioned in Section 2.1 and only keeps the gust regions. The resultant data was converted to binaries to facilitate the boundary tracing algorithm. The extracted and converted gust regions are shown in Figure 3, with gust regions in white. Note that no scale filters were applied to the patches presented in Figure 3.\\
Next, the Moore-Neighbor tracing algorithm with Jacob’s termination condition was applied to the binary data to trace the boundaries of the patches [14]. In the Moore-Neighbor tracing algorithm, an important concept is the Moore neighborhood which is a set of 8 pixels around a target pixel that share a vertex or edge with the target pixel. To track the boundary of a patch, as in Figure 3, the tracking algorithm uses any one of the white pixels on the boundary of the patch as the target pixel, and visits (moving clockwise for example) its Moore-neighbor pixels (black pixels) before entering another white pixel. Then, it uses the next white pixel (moving clockwise for example) as the target pixel and repeats the procedure. When it revisits the first white pixel it entered originally, the algorithm stops and all the visited black pixels comprise the boundary of the patch. There are two widely used termination conditions for the algorithm. The original one is to stop the algorithm after reentering the first white pixel for the second time. The other one is called Jacob’s stopping condition, which also stops the algorithm after reentering the first white pixel for the second time, but in the same direction one originally enters it. Since concave shapes are common for gust regions as seen in Figure 3, Jacob’s stopping condition was applied because it is more powerful to trace such shapes than the original stopping condition.\\
Moreover, since the patches with holes inside due to the above filtering still need to be considered as a whole, only exterior boundaries of the patches were traced. Additionally, patches with scale out of the range from 1D to 10D were filtered out and the centroids of the patches were calculated during the tracing process. The scale of a patch is defined as the square root of its area calculated by multiplying the grid area ($10~ m\times10~m$) by the number of the grid points inside or on the boundary. The retrieved wind field along with detected gust boundaries is shown in Figure 4.
\subsection{Tracking of gust patches}
To predict the future location of the gust patches, their advective characteristics need to be confirmed at first. Therefore, the next question is how to associate the patches between time frames given the detected regions. Given the formation of the spatial gust is contributed by the atmospheric turbulence, we assume that the gust regions propagate along the mean wind direction and their turbulent properties remain unchanged according to Taylor’s frozen turbulence hypothesis. Nevertheless, since the time intervals between two wind fields are integers, multiples of $30~s$, they are relatively long compared to the eddies in the atmosphere, so Taylor’s hypothesis is relaxed to some extent. It means slight changes in wind speed and wind direction, variations of scales, and deformation of the shapes are allowed in the proposed algorithm.\\
After extracting the gust regions from the wind field at two time frames, the tracking algorithm takes the detected gusts as inputs. The patches at an earlier frame are called “original patches,” and patches at later frame are called “candidate patches.” The time interval between the two frames should not be larger than $90~s$. Otherwise, the tracking algorithm will fail due to large changes in wind speed or shape caused by the evolution of the wind field.\\
Next, a searching zone is assigned to each original patch. In the searching zone, the algorithm searches for the corresponding target patch on the second frame over all the candidate patches. The candidate patches close to the boundaries of the studied wind field are neglected because of partial observation of the patches. The searching zone is placed downstream of the original patches and oriented along the mean wind direction of the measurement domain. The distance between the original patch and the center of the searching zone is set to the product of the mean wind speed of the measurement domain and the time interval between the frames. The size of the searching zone changes adaptively because a patch could deviate further from the streamline as time elapses. In that case, a large searching zone is needed to cover the possible area that the target patch could reach. The spanwise searching range is set proportional to the product of time and the standard deviation of wind velocity in the spanwise direction, and the same method is applied in the streamwise direction. The spanwise and streamwise velocity components can be calculated by projecting the retrieved x and y velocity components to the streamwise and spanwise directions. The searching zone constrains the tracking algorithm to a small window instead of the whole domain, which avoids unnecessary calculation. Examples of the original-target patch pairs along with their searching zones are presented in Figure 5c. The wind fields are shown in Figure 5 a,b.


