\chapter{Background}
\section{Wind gust detection and impact prediction for wind turbines}
Rapid changes of wind speed in the atmosphere, also called wind gusts, cause large fatigue loads on wind turbines. These loads reduce the lifetime of wind turbine components. Oscillations or ramping of the generated power can result in fast fluctuations of grid voltage and may pose additional burdens to the electric grid. Researchers have proposed adaptive and feed-forward control systems, which can adjust wind turbine settings for approaching winds [1–4]. A feed-forward control system requires accurate and fast gust detection system. Our purpose is to provide the information of wind gusts to the control systems.\\
There are a variety of gust detecting and tracking algorithms in the literature. The international standard IEC 61400-1 specifies standardization of several temporal gust models for wind turbine design. Branlard defined gusts as a short-term wind speed variation within a turbulent wind field [5,6]. Although different definitions exist, all suggest gusts invoke rapid wind speed changes. As many atmospheric sensors measure winds in relatively small volumes, changes in wind speed have been considered temporally. Temporal variations of wind speed can be converted to spatial variations using Taylor’s frozen turbulence hypothesis. Note that spatial variations in pressure on wind speed also damage buildings [7]. However, fewer studies directly measure spatial variations. The lack of such studies could be due to the limitation of the available instruments. Anemometers on met masts are the most common instruments for gust studies, but they are limited to measuring wind speed at fixed points. On the other hand, Doppler lidar can address this limitation. A long-range Doppler lidar can provide wind velocity field in a 3D domain up to 10 km with high temporal and spatial resolution.\\
The scale of a spatial gust is an important factor. Kelley et al. [8] and Chamorro et al. [9]. Sstudied the flow-structure interaction between wind turbines and atmospheric coherent structures with a scale ranging from the size of a wind turbine rotor to the thickness of the atmospheric boundary layer. They found that structures primarily in that range have a high correlation with the generated power and can induce strong structural responses. For structures with scales smaller than the size of a wind turbine rotor, the effects of their high-frequency components will be averaged out along the turbine blades and will not propagate to the drive train of a wind turbine and affect the power generation. For atmospheric gusts larger than 1000 m, varying winds in these large-scale gusts can be classified and captured as “meandering of the wind.” Effects of the wind meandering on wind turbines are complicated and begin to become relevant for yaw control. Therefore, we believe the gusts with scales between 100 m and 1000 m have the most significant effects on wind turbine performance.\\
In addition to the definition of wind gusts, the literature contains a variety of ideas for gust detecting and tracking. Mayor adapted two computer-vision methods for flow motion estimation: the cross-correlation method and the wavelet-based optical flow method [10,11]. However, the cross-correlation method has limitations for non-uniform velocity fields, and the optical flow method requires relatively small (few pixels) movement and is computationally demanding. These requirements make them impractical. On the other hand, Branlart [6] proposed several detection methods for different gust models, but gusts are defined in the time domain. He also estimated the arrival time of the gusts and presented an exponential probability distribution of the gust’s spanwise propagation. However, the gust distribution may not be representative, as the signal collected behind the turbine rotor would be heavily contaminated by the turbulence induced by the blades. The number of measurement points from anemometers is also limited. Therefore, a fast detecting and tracking algorithm and a reliable prediction model that can be used in real-time prediction of spatial gusts need to be developed.\\
In this work, we focus on spatial wind gusts within a limited range of scales. We propose a practical gust detecting and tracking algorithm with low computational cost. The novelty of the algorithm is to utilize dispersion and transport theory to create a practical tool which can provide short-range predictions of probable impact zones downstream for puffs or gusts detected upstream with a long-range Doppler lidar. The tool can provide real-time short-term predictions of impact time and location for gusts approaching a wind farm. The propagation of the wind gusts through the wind farm is not considered in this paper. By taking the gust size into consideration, the accuracy of the prediction can be increased, and valuable wind forecasting information can be provided to the control system.



\section{Time-of-Flight Lidar}
\subsection{Literature review of TOF lidar techniques}
Different lidar TOF measurement methods: direction TOF, indirect TOF, single return, multiple and avarge…, scanning/ flash …
Literature review of different lidar TOF measurements: direct (single shot(APD), multiple shot(GM-APD), AMCW, FMCW) , LED vs laser
CW vs pulsed --> this work focus only on pulsed lidar
\subsection{Literature review of TOF lidar simulation}
1. Importance of lidar simulator
2. Literature review of lidar simulators and their limitations
Physical-based lidar simulator : [Three-dimensional laser radar modelling
Ove Steinvall*, Tomas Carlsson Department], [Burns], [Budge, Adnan]
Game engine simulator[]

\section{Outline of the Dissertation}
