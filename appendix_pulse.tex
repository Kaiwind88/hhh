\chapter{Derivation of Relationship between Pulse Width and Rise Time}
%% derivation of sigma ~ FWHM and rise time
It can be derived from the rise time $t_r$ which is parameter people are more interested in. The rise time is defined as the time taken by a pulse to rise from 10\% to 90\% of its peak power. At the 90\% of the peak power, the instantaneous power is:
\begin{equation}
0.9P_0 = P_0e^{-\frac{-t_{90\%}^2}{2\sigma^2}}
\end{equation}
from which we can obtain
\begin{equation}
t_{90\%}=\sqrt{-2ln(0.9)} \approx0.459\sigma
\end{equation}
Similarly, we can obtain $t_{10\%} \approx2.146\sigma$ from
\begin{equation}
0.1P_0 = P_0e^{-\frac{-t_{10\%}^2}{2\sigma^2}}
\end{equation}
Thus, the rise time $t_r$ can be expressed by:
\begin{equation}
t_r = t_{10\%}-t_{90\%}\approx1.687\sigma
\end{equation}
Or
\begin{equation}
\sigma\approx\frac{t_r}{1.687}
\end{equation}
The pulse width of a laser pulse, which is usually defined as the Full Width Half Maximum (FWHM) of the pulse, can also be derived from Equation~\ref{eq: GauModel} :
\begin{equation}
0.5P_0 = P_0e^{-\frac{-t_{{\mathit{FWHM}}}^2}{2\sigma^2}}
\end{equation}
and
\begin{equation}
{\mathit{FWHM}}=2\sqrt{2ln2\sigma}\approx2.355\sigma
\end{equation}